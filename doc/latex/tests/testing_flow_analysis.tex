\documentclass[11pt,a4paper]{article}
\usepackage[utf8]{inputenc}
\usepackage[spanish]{babel}
\usepackage{amsmath,amssymb}
\usepackage{geometry}
\usepackage[hidelinks]{hyperref}
\usepackage{enumitem}
\usepackage{fancyhdr}
\usepackage{booktabs}
\usepackage{listings}
\usepackage{xcolor}

\geometry{margin=1in}
\pagestyle{fancy}
\fancyhf{}
\lhead{USP Testing Flow}
\rhead{v2.1.0 | 2026-02-20}
\cfoot{\thepage}

\lstset{
  language=bash,
  basicstyle=\ttfamily\small,
  keywordstyle=\color{blue},
  commentstyle=\color{gray},
  stringstyle=\color{red},
  breaklines=true,
  showstringspaces=false,
  tabsize=4,
  frame=single
}

\title{\textbf{Testing Flow Analysis}\\[0.5em]\large Universal Stochastic Predictor (v2.1.0-RC1)}
\author{Stochastic Predictor Team}
\date{20 de febrero de 2026}

\begin{document}

\maketitle
\thispagestyle{fancy}

\begin{abstract}
\noindent Este documento describe la arquitectura de testing reorganizada en dos capas de validación: \textbf{compliance} y \textbf{execution}. La pipeline utiliza auto-discovery dinámico para adaptarse a nuevos módulos sin hardcoding.
\end{abstract}

\tableofcontents
\newpage

\section{Executive Overview}

El sistema de testing ha sido reorganizado en una arquitectura modular de \textbf{2 capas de validación} orquestadas por un \textbf{entrypoint central} (\texttt{tests\_start.sh}). Cada capa valida un aspecto diferente del código:

\begin{center}
\begin{tabular}{|l|l|l|l|}
\hline
\textbf{Capa} & \textbf{Script} & \textbf{Propósito} & \textbf{Scope Discovery} \\
\hline
1. Compliance & \texttt{code\_alignement.py} & Valida 36 políticas & Auto-descubre módulos \\
\hline
2. Execution & \texttt{code\_structure.py} & Tests reales con pytest+JAX & Auto-descubre+parametriza \\
\hline
\end{tabular}
\end{center}

\vspace{1em}
\noindent\textbf{DEPRECATED}: \texttt{tests\_coverage.py} integrado en \texttt{code\_structure.py} (TestIOModuleImportable)

\section{Arquitectura Lógica}

\subsection{Pipeline de 2 Capas}

La arquitectura sigue un flujo secuencial con estrategia \textit{fail-fast}:

\begin{enumerate}[leftmargin=2cm]
    \item[\textbf{Entrypoint}] \texttt{tests\_start.sh} - Orchestrator principal (Bash)
    
    \item[\textbf{Layer 1}] \texttt{code\_alignement.py} - Policy Compliance Checker
    \begin{itemize}[noitemsep]
        \item Valida 36 CODE\_AUDIT\_POLICIES
        \item Scope: Todo el repositorio
        \item Auto-descubre módulos Python
        \item Output: JSON + Markdown reports
        \item \textbf{FAIL $\to$ EXIT} (detiene pipeline)
    \end{itemize}
    
    \item[\textbf{Layer 2}] \texttt{code\_structure.py} - Code Execution Tests
    \begin{itemize}[noitemsep]
        \item Ejecuta tests reales con pytest + JAX
        \item Auto-genera tests parametrizados para io/
        \item Coverage: 126 funciones en api, core, kernels, io
        \item Output: JSON + Markdown reports
    \end{itemize}
\end{enumerate}

\subsection{Reportes Generados}

Cada script genera dos tipos de reportes:

\begin{center}
\begin{tabular}{|l|l|l|}
\hline
\textbf{Script} & \textbf{JSON Report} & \textbf{Markdown Report} \\
\hline
code\_alignement.py & tests/results/code\_alignement\_last.json & tests/reports/code\_alignement\_last.md \\
\hline
code\_structure.py & tests/results/code\_structure\_last.json & tests/reports/code\_structure\_last.md \\
\hline
\end{tabular}
\end{center}

\noindent\textbf{Nota}: Los archivos usan sufijo \texttt{\_last} (sin timestamp) para facilitar acceso al último resultado.

\section{Scope Discovery}

\subsection{Módulo Central: scope\_discovery.py}

Todas las funciones de auto-descubrimiento están centralizadas en \texttt{tests/scripts/scope\_discovery.py}:

\begin{lstlisting}[language=Python]
def discover_modules(root) -> List[str]:
    """Auto-discover all submodules in Python/ directory."""
    # Returns: ['api', 'core', 'io', 'kernels']
    
def extract_public_api(module_name) -> Set[str]:
    """Extract __all__ exports from module __init__.py"""
    
def discover_all_public_api() -> Dict[str, Set[str]]:
    """Map all modules to their public symbols"""
\end{lstlisting}

\subsection{Ventajas del Auto-Discovery}

\begin{itemize}[noitemsep]
    \item \textbf{Cero hardcoding}: No listas manuales de módulos/funciones
    \item \textbf{Auto-adaptación}: Detecta nuevos módulos automáticamente
    \item \textbf{Mantenimiento reducido}: Sin actualización manual de scopes
    \item \textbf{Consistencia}: Misma lógica de discovery en todos los scripts
\end{itemize}

\section{Scripts Detallados}

\subsection{Stage 1: code\_alignement.py (Policy Compliance)}

\textbf{Propósito}: Validar 36 políticas de auditoría de código.

\textbf{Scope}: Repositorio completo (no limitado a Python/)

\textbf{Políticas Validadas}:
\begin{itemize}[noitemsep]
    \item Configuration sourcing (zero-heuristics)
    \item Configuration immutability (locked subsections)
    \item Validation schema enforcement
    \item Atomic configuration mutation protocol
    \item 64-bit precision enablement
    \item Stop-gradient for diagnostics
    \item Kernel purity and statelessness
    \item Five-layer architecture enforcement
    \item Dependency pinning (exact versions)
    \item \textit{...y 27 políticas adicionales}
\end{itemize}

\textbf{Output}:
\begin{itemize}[noitemsep]
    \item Console: PASS/FAIL por política
    \item JSON: \texttt{tests/results/code\_alignement\_last.json}
    \item Markdown: \texttt{tests/reports/code\_alignement\_last.md}
\end{itemize}

\subsection{Stage 2: code\_structure.py (Execution Tests)}

\textbf{Propósito}: Ejecutar tests reales con JAX y pytest.

\textbf{Scope}: Python/* (todos los submódulos)

\textbf{Estrategia de Testing}:
\begin{itemize}[noitemsep]
    \item Tests explícitos para api, core, kernels (20+ test classes)
    \item Tests auto-parametrizados para io/ (via TestIOModuleImportable)
    \item Coverage: 126 funciones (api:53, core:20, kernels:23, io:30)
    \item Ejecución real con JAX x64 habilitado
\end{itemize}

\textbf{Módulos Cubiertos}:
\begin{center}
\begin{tabular}{|l|r|l|}
\hline
\textbf{Módulo} & \textbf{Funciones} & \textbf{Test Strategy} \\
\hline
api & 53 & Explicit test classes \\
\hline
core & 20 & Explicit test classes \\
\hline
kernels & 23 & Explicit test classes \\
\hline
io & 30 & Auto-parametrized (TestIOModuleImportable) \\
\hline
\textbf{Total} & \textbf{126} & \textbf{Mixed} \\
\hline
\end{tabular}
\end{center}

\section{Ejecución de Tests}

\subsection{Comando Principal}

\begin{lstlisting}[language=bash]
# Ejecutar toda la pipeline (compliance + execution)
./tests/scripts/tests_start.sh --all

# Solo compliance
./tests/scripts/tests_start.sh --compliance

# Solo execution tests
./tests/scripts/tests_start.sh --execute
\end{lstlisting}

\subsection{Flujo de Ejecución}

\begin{enumerate}
    \item \textbf{Validación del entorno}
    \begin{itemize}[noitemsep]
        \item Verifica existencia de \texttt{.venv/bin/python}
        \item Valida directorios \texttt{tests/results} y \texttt{tests/reports}
    \end{itemize}
    
    \item \textbf{Stage 1: Compliance Check}
    \begin{itemize}[noitemsep]
        \item Ejecuta \texttt{code\_alignement.py}
        \item Valida 36 políticas
        \item Si FAIL $\to$ detiene pipeline (fail-fast)
    \end{itemize}
    
    \item \textbf{Stage 2: Execution Tests}
    \begin{itemize}[noitemsep]
        \item Ejecuta \texttt{code\_structure.py} vía pytest
        \item Tests reales con JAX
        \item Genera reportes JSON + Markdown
    \end{itemize}
    
    \item \textbf{Summary}
    \begin{itemize}[noitemsep]
        \item Total/Passed/Failed por stage
        \item Lista últimos artefactos en \texttt{tests/results/}
    \end{itemize}
\end{enumerate}

\section{Troubleshooting}

\subsection{Errores Comunes}

\textbf{Error: Python virtual environment not found}
\begin{lstlisting}[language=bash]
# Solución: Crear y activar .venv
python3 -m venv .venv
source .venv/bin/activate
pip install -r requirements.txt
\end{lstlisting}

\textbf{Error: Compliance check failed}
\begin{itemize}[noitemsep]
    \item Revisar \texttt{tests/reports/code\_alignement\_last.md}
    \item Identificar qué política falló
    \item Corregir código según especificación en \texttt{doc/latex/tests/code\_audit\_policies.tex}
\end{itemize}

\textbf{Error: JAX not installed}
\begin{lstlisting}[language=bash]
# Solución: Instalar dependencias
source .venv/bin/activate
pip install -r requirements.txt
\end{lstlisting}

\subsection{Verificación Manual}

\begin{lstlisting}[language=bash]
# Verificar descubrimiento de módulos
cd tests/scripts
python3 -c "from scope_discovery import discover_modules; print(discover_modules())"
# Expected: ['api', 'core', 'io', 'kernels']

# Verificar API pública de un módulo
python3 -c "from scope_discovery import extract_public_api; print(extract_public_api('api'))"

# Ejecutar compliance manualmente
.venv/bin/python tests/scripts/code_alignement.py

# Ejecutar structure tests manualmente
.venv/bin/python tests/scripts/code_structure.py
\end{lstlisting}

\section{Changelog: 3-Layer $\to$ 2-Layer}

\subsection{Arquitectura Anterior (3 Capas)}

\begin{enumerate}
    \item \textbf{code\_alignement.py} - Policy compliance
    \item \textbf{tests\_coverage.py} - Coverage validator
    \item \textbf{code\_structure.py} - Execution tests
\end{enumerate}

\subsection{Problema Identificado}

Si \texttt{code\_structure.py} auto-descubre funciones y las testea, entonces \texttt{tests\_coverage.py} siempre pasará (redundancia lógica).

\subsection{Solución: Consolidación}

\begin{itemize}[noitemsep]
    \item \textbf{Eliminado}: \texttt{tests\_coverage.py} (marcado como DEPRECATED)
    \item \textbf{Consolidado}: Lógica de coverage en \texttt{code\_structure.py} (TestIOModuleImportable)
    \item \textbf{Resultado}: Pipeline más simple y sin redundancia
\end{itemize}

\subsection{Commits Relevantes}

\begin{lstlisting}[language=bash]
# feat: create scope_discovery.py with auto-discovery functions
# refactor: use dynamic paths in code_alignement.py (38+ replacements)
# refactor: consolidate to 2-layer pipeline (remove coverage stage)
# feat: add TestIOModuleImportable for Python/io/ coverage
# docs: create LaTeX documentation at doc/latex/tests/
\end{lstlisting}

\section{Referencias}

\subsection{Documentación Relacionada}

\begin{itemize}[noitemsep]
    \item \texttt{doc/latex/tests/code\_audit\_policies.tex} - 36 políticas de código
    \item \texttt{doc/latex/tests/testing\_audit\_policies.tex} - 45 políticas de testing
    \item \texttt{tests/scripts/scope\_discovery.py} - Módulo de auto-discovery
    \item \texttt{tests/scripts/code\_alignement.py} - Implementación de políticas
    \item \texttt{tests/scripts/code\_structure.py} - Suite de tests de ejecución
\end{itemize}

\subsection{Especificación Matemática}

\begin{itemize}[noitemsep]
    \item \texttt{doc/latex/specification/Stochastic\_Predictor\_Test\_Cases.tex}
    \item \texttt{doc/latex/specification/Stochastic\_Predictor\_Tests\_Python.tex}
\end{itemize}

\vspace{2em}
\noindent\textbf{Document Approved For:} Implementation \\
\textbf{Last Updated:} 2026-02-20 \\
\textbf{Maintainer:} Stochastic Predictor Team

\end{document}
