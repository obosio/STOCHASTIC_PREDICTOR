\documentclass[11pt, a4paper]{report}

% --- PREAMBLE ---
\usepackage[a4paper, top=2.5cm, bottom=2.5cm, left=2cm, right=2cm]{geometry}
\usepackage{fontspec}
\usepackage{amsmath}
\usepackage{amssymb}
\usepackage{amsfonts}
\usepackage{mathtools}
\usepackage{booktabs}
\usepackage{enumitem}
\usepackage{amsthm}

\usepackage[english]{babel}

% Environments
\newtheorem{definition}{Definition}[chapter]
\newtheorem{testcase}{Test Case}[chapter]
\newtheorem{criterion}{Criterion}[chapter]
\newtheorem{remark}{Note}[chapter]

% --- HYPERREF (must be last) ---
\usepackage[hidelinks]{hyperref}

\title{\textbf{Validation and Test Protocol \\ for the Universal Stochastic Predictor}}
\author{Adaptive Meta-Prediction Development Consortium}
\date{\today}

\begin{document}

\maketitle

\tableofcontents

\chapter{Unit Tests: Kernels and Fundamental Algorithms}

These tests verify isolated implementation of critical algorithms without relying on global system state.

\section{Entropy and Random Variable Generation}

\subsection{CMS Algorithm}

\begin{testcase}[Validation of $\alpha$-Stable Distributions]
Validate that generating $\alpha$-stable variables via the Chambers-Mallows-Stuck method produces distributions with the desired parameters $(\alpha, \beta, \gamma, \delta)$.
\end{testcase}

\begin{criterion}
For a sample size $N \geq 10^4$, empirical moments must match the theoretical properties of the stable distribution within a 95\% confidence interval.
\end{criterion}

\subsection{Pseudo-Random Generator Integrity}

\begin{testcase}[Mersenne Twister/PCG64]
Verify the absence of serial correlations and the long period guarantees of the pseudo-random number generator.
\end{testcase}

\begin{criterion}
Apply standard statistical test batteries (TestU01, Diehard) and verify that no randomness tests fail. The generator period must be $\geq 2^{127}$ for Mersenne Twister and $\geq 2^{128}$ for PCG64.
\end{criterion}

\section{Singularity Analysis (SIA)}

\subsection{Holder Exponent Detection}

\begin{testcase}[WTMM Validation]
Use synthetic signals with known Holder exponent $H$ to validate that the WTMM (Wavelet Transform Modulus Maxima) algorithm recovers the singularity spectrum $D(h)$ with error $< 5\%$.
\end{testcase}

\begin{criterion}
Let $f(t)$ be a synthetic signal with known $H = H_0$. The Holder exponent estimated by WTMM must satisfy:
\[
|\hat{H} - H_0| < 0.05 \cdot H_0
\]
where $\hat{H}$ is the estimate from multiscale analysis.
\end{criterion}

\subsection{Cone of Influence}

\begin{testcase}[Besov Influence Radius]
Verify that maxima linking in the scale space respects the influence radius defined by $C_{besov}$.
\end{testcase}

\begin{criterion}
For two consecutive maxima at scales $s_1 < s_2$, their temporal positions $(t_1, t_2)$ must satisfy:
\[
|t_2 - t_1| \leq C_{besov} \cdot (s_2 - s_1)
\]
where $C_{besov}$ is the Besov-space constant associated with the chosen wavelet.
\end{criterion}

\subsection{Soft Nyquist Limit Validation}

The I/O specification defines a minimum data injection frequency to preserve multifractal analysis integrity. This test explicitly validates that operational limit.

\begin{testcase}[Multifractal Aliasing]
Gradually reduce the input sampling frequency until the spectrum $D(h)$ collapses, validating that the system detects the undersampling condition before irreversible degradation occurs.
\end{testcase}

\begin{criterion}
Consider a reference multifractal signal with known singularity spectrum $D_0(h)$ and nominal sampling frequency $f_0$. Use the following protocol:
\begin{enumerate}
    \item Reduce sampling frequency by decimation: $f_k = f_0 / 2^k$ with $k = 0, 1, 2, \ldots$
    \item For each $f_k$, compute the estimated spectrum $\hat{D}_k(h)$ via WTMM
    \item Compute relative error:
    \[
    \varepsilon_k = \frac{\|D_0(h) - \hat{D}_k(h)\|_{L^2}}{\|D_0(h)\|_{L^2}}
    \]
    \item Monitor the dominant Holder exponent: $\hat{H}_k = \arg\max_h D_k(h)$
\end{enumerate}
The system must satisfy:
\begin{itemize}
    \item \textbf{Preventive detection:} Emit \texttt{FreezingTopologicalBranchEvent} when $\varepsilon_k > 0.05$ (error $> 5\%$)
    \item \textbf{Acceptance criterion:} The signal must trigger \textit{before} Holder error exceeds 10\%:
    \[
    \frac{|\hat{H}_k - H_0|}{H_0} < 0.10 \quad \text{at the moment of freezing}
    \]
    \item \textbf{Corrective action:} After the signal, freeze the topological branch weight at its last reliable value: $w_C \to w_C^{frozen}$ with no dynamic updates
\end{itemize}
\end{criterion}

\begin{remark}
The Nyquist limit in multifractal analysis is a soft boundary. Self-similarity allows some tolerance to undersampling, but below a threshold fine-scale structures collapse and $D(h)$ degrades. This test ensures the system operates in a safe resolution regime.
\end{remark}

\begin{remark}
The critical frequency $f_{critical}$ depends on:
\begin{enumerate}
    \item The scale range $[s_{min}, s_{max}]$ of the wavelet transform
    \item The temporal support of the mother wavelet $\psi(t)$
    \item The expected minimum Holder regularity $H_{min}$
\end{enumerate}
A practical heuristic:
\[
f_{min} \geq \frac{10}{s_{min}} \cdot (1 + H_{min}^{-1})
\]
This ensures at least 10 points per minimum scale, adjusted by roughness.
\end{remark}

\section{Algebraic Structures (Branch D)}

\subsection{Chen Identity}

\begin{testcase}[Signature Concatenation]
Validate that concatenating signatures of two path segments via the tensor product ($\otimes$) equals the signature of the full path.
\end{testcase}

\begin{criterion}
Let $\gamma_1: [0, T_1] \to \mathbb{R}^d$ and $\gamma_2: [0, T_2] \to \mathbb{R}^d$ be two paths. Chen's identity states:
\[
\text{Sig}(\gamma_1 \star \gamma_2) = \text{Sig}(\gamma_1) \otimes \text{Sig}(\gamma_2)
\]
where $\gamma_1 \star \gamma_2$ denotes concatenation. Numerical error must be $< 10^{-6}$ in Euclidean norm.
\end{criterion}

\subsection{Scale Invariance}

\begin{testcase}[Temporal Reparametrization]
Check that the level-$M$ truncated signature is invariant under strictly increasing time reparametrizations of the input path.
\end{testcase}

\begin{criterion}
For a strictly increasing reparametrization $\phi: [0,1] \to [0,1]$ with $\phi(0)=0$ and $\phi(1)=1$, we must have:
\[
\text{Sig}^{(M)}(\gamma) = \text{Sig}^{(M)}(\gamma \circ \phi)
\]
where $\text{Sig}^{(M)}$ is the level-$M$ truncated signature.
\end{criterion}

\chapter{Integration Tests and Stochastic Convergence}

Validate that component interactions respect continuous probability laws and numerical stability conditions.

\section{Stochastic Differential Equation (SDE) Solvers}

\subsection{Convergence of Numerical Schemes}

\begin{testcase}[Euler-Maruyama vs Milstein]
For diffusion with non-constant volatility, verify that Milstein achieves strong convergence order 1.0 versus order 0.5 for Euler-Maruyama.
\end{testcase}

\begin{criterion}
Consider the SDE:
\[
 dX_t = \mu(X_t, t)\, dt + \sigma(X_t, t)\, dW_t
\]
with non-constant $\sigma(x,t)$. For a sequence of time steps $\Delta t_n = 2^{-n} \Delta t_0$, the strong error must satisfy:
\begin{align*}
E[|X_T - X_T^{EM}|^2] &= O(\Delta t^{0.5}) \quad \text{(Euler-Maruyama)} \\
E[|X_T - X_T^{M}|^2] &= O(\Delta t^{1.0}) \quad \text{(Milstein)}
\end{align*}
where $X_T^{EM}$ and $X_T^{M}$ are the numerical approximations.
\end{criterion}

\subsection{Mixed CFL Condition}

\begin{testcase}[Stability Violation]
Force a time step $\Delta t$ that violates the Courant-Friedrichs-Lewy restriction and confirm numerical instability or NaNs as expected behavior to calibrate the safety monitor.
\end{testcase}

\begin{criterion}
For an explicit scheme, the CFL condition requires:
\[
\Delta t \leq \frac{C}{\sup_x |\mu'(x)| + \sigma^2(x)/2}
\]
Force $\Delta t > 10 C$ and verify divergence detection via:
\begin{itemize}
    \item NaN or Inf emergence in simulated trajectories
    \item Instability alert emitted by the safety module
\end{itemize}
\end{criterion}

\section{Transport Optimization (Orchestrator)}

\subsection{Sinkhorn Algorithm Stability}

\begin{testcase}[Log-Domain Convergence]
Evaluate Sinkhorn convergence in the log domain with decreasing regularization $\varepsilon$, ensuring no underflow down to $\varepsilon \geq 10^{-4}$.
\end{testcase}

\begin{criterion}
The log-domain Sinkhorn-Knopp algorithm must converge when:
\[
\varepsilon \geq 10^{-4}
\]
For $\varepsilon < 10^{-4}$, the system must detect underflow risk and emit a warning. Convergence is measured by:
\[
\|K \text{diag}(u) K^T \text{diag}(v) - \mu\|_1 < 10^{-6}
\]
where $K_{ij} = \exp(-C_{ij}/\varepsilon)$ is the Gibbs kernel.
\end{criterion}

\subsection{Simplex Normalization}

\begin{testcase}[Probabilistic Mass Conservation]
Confirm that after any JKO update the sum of kernel weights is strictly $\sum_i \rho_i = 1.0$.
\end{testcase}

\begin{criterion}
After each JKO iteration:
\[
\rho^{n+1} = \arg\min_{\rho \in \mathcal{P}(\mathcal{X})} \left\{ W_2^2(\rho, \rho^n) + \tau \mathcal{F}[\rho] \right\}
\]
verify that $\rho^{n+1}$ belongs to the probability simplex:
\[
\sum_{i=1}^{N} \rho_i^{n+1} = 1.0, \quad \rho_i^{n+1} \geq 0 \quad \forall i
\]
with numerical tolerance $|\sum_i \rho_i^{n+1} - 1.0| < 10^{-10}$.
\end{criterion}

\section{HJB Solution via DGM (Branch B)}

Branch B solves the Hamilton-Jacobi-Bellman equation with Deep Galerkin Method (DGM). These tests validate convergence and stability.

\subsection{Gradient Stability}

\begin{testcase}[Gradient Explosion Under High Volatility]
Monitor gradient norms during PDE training to detect gradient explosions in high-volatility regimes.
\end{testcase}

\begin{criterion}
The HJB equation:
\[
\frac{\partial V}{\partial t} + \sup_{u \in \mathcal{U}} \left\{ \mathcal{L}^u V(x,t) + f(x,u,t) \right\} = 0
\]
is approximated by a neural network $V_\theta(x,t)$ trained with DGM. During training, monitor:
\begin{enumerate}
    \item Loss gradient norm:
    \[
    \|\nabla_\theta \mathcal{L}_{DGM}(\theta)\|_2 = \left\| \frac{\partial}{\partial \theta} \mathbb{E}[|PDE[V_\theta]|^2 + |BC[V_\theta]|^2] \right\|_2
    \]
    \item Apply gradient clipping when $\|\nabla_\theta \mathcal{L}\|_2 > C_{clip}$ with $C_{clip} = 10.0$
    \item In high-volatility scenarios ($\sigma(x,t) > 2\sigma_0$), the system must:
    \begin{itemize}
        \item Detect if $\|\nabla_\theta \mathcal{L}\|_2$ exceeds $C_{clip}$ for more than 5 consecutive iterations
        \item Emit \texttt{GradientInstabilityEvent}
        \item Adaptively reduce learning rate: $\eta \to 0.5 \eta$
        \item Verify stabilization within the next 20 epochs
    \end{itemize}
\end{enumerate}
\end{criterion}

\begin{remark}
Gradient explosions in DGM often arise from interactions between second-order derivatives and diffusion coefficients. Early detection and adaptive clipping are essential for convergence in high-volatility regimes.
\end{remark}

\subsection{Viscosity Solution Validation}

\begin{testcase}[Crandall-Lions Comparison Principle]
Validate that the neural solution $V_\theta(x,t)$ respects the comparison principle for viscosity solutions, ensuring uniqueness and consistency for non-linear PDEs.
\end{testcase}

\begin{criterion}
The comparison principle states: if $u$ is a viscosity supersolution and $v$ is a viscosity subsolution of the same HJB equation with $u \geq v$ on the boundary, then $u \geq v$ on the full domain.

Validation procedure:
\begin{enumerate}
    \item Construct a reference solution $V_{ref}(x,t)$ using a standard method (upwind finite differences, method of lines)
    \item Train the DGM network to obtain $V_\theta(x,t)$
    \item Verify time monotonicity (finite horizon):
    \[
    V_\theta(x, t_1) \geq V_\theta(x, t_2) \quad \forall t_1 < t_2, \forall x \in \mathcal{D}
    \]
    (assuming non-negative costs)
    \item Validate sub/supersolution constraints on a test grid $\{(x_i, t_j)\}_{i,j}$:
    \begin{align*}
    \text{Subsolution:} \quad & PDE[V_\theta](x_i, t_j) \leq \epsilon_{tol} \\
    \text{Supersolution:} \quad & PDE[V_\theta](x_i, t_j) \geq -\epsilon_{tol}
    \end{align*}
    with $\epsilon_{tol} = 10^{-3}$
    \item Compare with reference solution:
    \[
    \|V_\theta - V_{ref}\|_{L^\infty(\mathcal{D})} < \delta_{viscosity}
    \]
    where $\delta_{viscosity} = 0.05 \cdot \|V_{ref}\|_{L^\infty}$ (relative error $< 5\%$)
\end{enumerate}
\end{criterion}

\begin{remark}
Crandall-Lions viscosity theory is the rigorous framework for HJB equations. DGM networks, being smooth functions, must approximate these generalized solutions. The comparison principle test prevents spurious oscillations and causality violations.
\end{remark}

\subsection{Training Entropy Test (Mode Collapse)}

During neural PDE training, the model can collapse to trivial constant solutions. This test detects and prevents that behavior.

\begin{testcase}[Mode Collapse Detection]
Verify that the neural network does not collapse to a trivial solution that satisfies the PDE but fails to capture structure.
\end{testcase}

\begin{criterion}
Mode collapse occurs when the network learns a minimum-variance solution that minimizes PDE loss without solving the boundary-value problem.

Detection protocol:
\begin{enumerate}
    \item Compute spatial variance at time $t < T$:
    \[
    \text{Var}_x[V_\theta(x,t)] = \mathbb{E}_x[(V_\theta(x,t) - \bar{V}_t)^2]
    \]
    where $\bar{V}_t = \mathbb{E}_x[V_\theta(x,t)]$
    \item Compute variance of the terminal condition:
    \[
    \text{Var}[g(\xi)] = \mathbb{E}_\xi[(g(\xi) - \bar{g})^2]
    \]
    \item Verify proportionality:
    \[
    \kappa_{low} \leq \frac{\text{Var}_x[V_\theta(x,t)]}{\text{Var}[g(\xi)]} \leq \kappa_{high}
    \]
    with typical thresholds $\kappa_{low} = 0.3$ and $\kappa_{high} = 1.2$
    \item If the ratio falls below $\kappa_{low}$: detect mode collapse
    \item Monitor differential entropy:
    \[
    H[V_\theta] = -\int p(v) \log p(v) \, dv
    \]
    A collapsed solution yields $H[V_\theta] \to -\infty$ (delta distribution)
\end{enumerate}

Acceptance criteria:
\begin{itemize}
    \item Variance ratio in $[\kappa_{low}, \kappa_{high}]$ for at least 90\% of $t \in [0,T]$
    \item Differential entropy satisfies $H[V_\theta] > H_{min}$
    \item Spatial gradient norm does not collapse:
    \[
    \mathbb{E}_x[\|\nabla_x V_\theta(x,t)\|_2] > \epsilon_{grad} > 0
    \]
\end{itemize}
\end{criterion}

\begin{remark}
Mode collapse is common in:
\begin{enumerate}
    \item Uniform or symmetric boundary conditions
    \item Excessive learning rates leading to minimum-norm solutions
    \item Under-parameterized networks
    \item Biased weight initialization
\end{enumerate}
Early variance monitoring allows reinitialization or hyperparameter tuning before collapse is permanent.
\end{remark}

\begin{remark}
From a control theory perspective, a collapsed solution $V_\theta(x,t) \approx c$ yields a degenerate control policy $u^*(x,t)$ that does not respond to state changes, which is operationally useless.
\end{remark}

\subsection{Mesh Refinement Convergence}

\begin{testcase}[Consistency under Refinement]
Verify that the DGM solution converges to the exact (or reference) solution as the density of collocation points increases.
\end{testcase}

\begin{criterion}
For nested meshes $\mathcal{M}_1 \subset \mathcal{M}_2 \subset \mathcal{M}_3$ with densities $N_1 < N_2 < N_3$:
\begin{enumerate}
    \item Train DGM on each mesh to loss convergence: $\mathcal{L}_{DGM}^{(k)} < 10^{-4}$
    \item Compute error on a fixed fine evaluation mesh:
    \[
    e_k = \|V_{\theta_k} - V_{ref}\|_{L^2(\mathcal{D}_{eval})}
    \]
    \item Verify monotone convergence:
    \[
    e_1 > e_2 > e_3
    \]
    \item Estimate empirical rate:
    \[
    r = \frac{\log(e_1/e_2)}{\log(N_2/N_1)}
    \]
    Expected $r \geq 0.5$ (at least order $N^{-0.5}$ convergence)
\end{enumerate}
\end{criterion}

\subsection{Simplified Output Variance Test (Mode Collapse)}

This test complements entropy monitoring with a direct variance ratio check.

\begin{testcase}[Minimum Variance Threshold]
Compare output variance of the neural solution $V_\theta(x,t)$ against a reference solution $V_{ref}(x,t)$ and verify that the network captures at least 10\% of the true variability.
\end{testcase}

\begin{criterion}
Let $V_{ref}(x,t)$ be a reference solution obtained with a standard method. For the trained DGM network $V_\theta(x,t)$:
\begin{enumerate}
    \item Compute reference variance at time $t \in [0,T]$:
    \[
    \text{Var}_{ref}(t) = \frac{1}{|\mathcal{X}|} \sum_{x_i \in \mathcal{X}} (V_{ref}(x_i, t) - \bar{V}_{ref}(t))^2
    \]
    \item Compute neural variance:
    \[
    \text{Var}_{\theta}(t) = \frac{1}{|\mathcal{X}|} \sum_{x_i \in \mathcal{X}} (V_\theta(x_i, t) - \bar{V}_\theta(t))^2
    \]
    \item Variance ratio:
    \[
    R_{var}(t) = \frac{\text{Var}_{\theta}(t)}{\text{Var}_{ref}(t)}
    \]
    \item \textbf{Failure criterion:}
    \[
    R_{var}(t) < 0.10 \quad \text{for any } t \in [0, 0.9T]
    \]
    \item \textbf{Acceptance criterion:}
    \[
    R_{var}(t) \geq 0.10 \quad \forall t \in [0, 0.9T]
    \]
    and
    \[
    \text{median}_{t \in [0,T]} R_{var}(t) \geq 0.50
    \]
\end{enumerate}
\end{criterion}

\begin{remark}
If $R_{var}(t) < 0.10$ before convergence, interrupt training and adjust hyperparameters (learning rate, architecture, initialization). The 10\% threshold is conservative; any solution below it is effectively constant. The $t \in [0, 0.9T]$ restriction excludes the terminal region where variance naturally contracts.
\end{remark}

\begin{remark}
This criterion complements the entropy test. Practical interpretation:
\begin{itemize}
    \item $R_{var} < 0.10$: critical collapse (fail)
    \item $0.10 \leq R_{var} < 0.30$: low-variance warning
    \item $R_{var} \geq 0.50$: normal operation
\end{itemize}
\end{remark}

\chapter{Robustness Tests and Circuit Breakers}

These tests verify system protection against market anomalies or data failures.

\section{Outlier and Regime Handling}

\subsection{Outlier Injection}

\begin{testcase}[Extreme Outlier Values]
Inject values $> 20\sigma$ in the input stream and verify that the system rejects the point, emits a validation alert, and preserves inertial weights.
\end{testcase}

\begin{criterion}
Given observations $\{y_t\}$ with rolling mean $\mu_t$ and standard deviation $\sigma_t$, inject:
\[
\tilde{y}_t = \mu_t + 20\sigma_t
\]
The system must:
\begin{enumerate}
    \item Detect $|\tilde{y}_t - \mu_t| > \theta\sigma_t$ with $\theta = 10$
    \item Reject the observation and NOT update the meta-state $\Xi_t$
    \item Emit \texttt{OutlierDetectedEvent} with metadata for the rejected value
    \item Keep weights $\{w_i\}_{i=A}^D$ unchanged
\end{enumerate}
\end{criterion}

\subsection{CUSUM Trigger}

\begin{testcase}[Structural Regime Change]
Simulate a regime change (structural drift) and validate that the change event is emitted exactly when $G_t^+$ exceeds the dynamic threshold $h$.
\end{testcase}

\begin{criterion}
CUSUM detects mean shifts via:
\[
G_t^+ = \max(0, G_{t-1}^+ + (y_t - \mu_0) - k)
\]
where $k$ is the slack and $h$ is the alarm threshold. Verify:
\begin{enumerate}
    \item $G_t^+ > h \Rightarrow$ emit \texttt{RegimeChangedEvent}
    \item Reset $G_t^+ = 0$ after detection
    \item Detection delay is at most $\tau$ observations after the true change point
\end{enumerate}
\end{criterion}

\section{Emergency Mode (Robustness Postulate)}

\subsection{Critical Singularity}

\begin{testcase}[Extreme Roughness Regime]
Artificially reduce the Holder exponent below $H_{min}$ and verify that the orchestrator forces $w_D \to 1.0$ and switches the cost to the Huber metric.
\end{testcase}

\begin{criterion}
Define a critical threshold $H_{min}$ (typically $H_{min} = 0.25$). When WTMM detects:
\[
\hat{H}_t < H_{min}
\]
The system must:
\begin{enumerate}
    \item Activate emergency mode: $w_A = w_B = w_C = 0$, $w_D = 1.0$
    \item Switch orchestrator cost from Wasserstein to Huber:
    \[
    C(x,y) = \begin{cases}
    \frac{1}{2}|x-y|^2 & \text{if } |x-y| \leq \delta \\
    \delta(|x-y| - \frac{\delta}{2}) & \text{if } |x-y| > \delta
    \end{cases}
    \]
    \item Emit \texttt{CriticalSingularityEvent}
    \item Maintain this state until $\hat{H}_t > H_{min} + \epsilon_{hysteresis}$
\end{enumerate}
\end{criterion}

\chapter{I/O and Persistence Tests}

These tests guarantee operational continuity and latent state integrity.

\section{Snapshot Protocol}

\subsection{Hot-Start}

\begin{testcase}[State Continuity]
Serialize the meta-state $\Xi_t$, restart the system, and load it. The first prediction after restart must match the prediction without interruption.
\end{testcase}

\begin{criterion}
Define the full meta-state:
\[
\Xi_t = \left\{ \{w_i\}_{i=A}^D, \{\theta_i^*\}_{i=A}^D, \mathcal{H}_t, \text{Sig}_t, G_t^{\pm}, \mu_t, \sigma_t^2 \right\}
\]
Validation procedure:
\begin{enumerate}
    \item At time $t_0$, serialize $\Xi_{t_0}$ to a binary file
    \item Generate prediction $\hat{y}_{t_0+1}^{\text{original}}$
    \item Restart the system (release memory)
    \item Load $\Xi_{t_0}$ from file
    \item Generate prediction $\hat{y}_{t_0+1}^{\text{restored}}$
    \item Verify: $|\hat{y}_{t_0+1}^{\text{original}} - \hat{y}_{t_0+1}^{\text{restored}}| < 10^{-12}$
\end{enumerate}
\end{criterion}

\subsection{Checksum Validation}

\begin{testcase}[Cryptographic Integrity]
Corrupt a single bit in the snapshot file and verify that the system rejects the load via SHA-256, forcing a cold start.
\end{testcase}

\begin{criterion}
Each snapshot must include a SHA-256 hash. Load must:
\begin{enumerate}
    \item Read the binary file
    \item Compute $H' = \text{SHA256}(\text{content})$
    \item Compare to stored hash $H$
    \item If $H' \neq H$: reject load, emit \texttt{CorruptedSnapshotEvent}, initialize cold start
    \item If $H' = H$: proceed with deserialization
\end{enumerate}
Validate by flipping one random bit and verifying rejection.
\end{criterion}

\section{I/O Failure Recovery}

Persisting $\Xi_t$ is critical for continuity. These tests validate robustness against write, read, and storage failures.

\subsection{Write Interruption (Atomicity)}

\begin{testcase}[Power Loss Simulation]
Simulate a sudden power loss during snapshot serialization and verify that partially written files are handled safely.
\end{testcase}

\begin{criterion}
Snapshotting must guarantee atomicity via write-then-rename:
\begin{enumerate}
    \item Serialize $\Xi_t$ to temporary file: \texttt{snapshot\_\{timestamp\}.tmp}
    \item Compute $H = \text{SHA256}(\Xi_t)$ and append to file
    \item Perform \texttt{fsync()} to flush to disk
    \item Atomically rename: \texttt{snapshot\_\{timestamp\}.tmp} $\to$ \texttt{snapshot\_\{timestamp\}.bin}
\end{enumerate}
Validation:
\begin{enumerate}
    \item Start snapshot at time $t_0$
    \item Interrupt at random progress $p \in \{0.1, 0.3, 0.5, 0.7, 0.9\}$
    \item Simulate abrupt termination (\texttt{SIGKILL} or I/O cut)
    \item Restart system
    \item The system must:
    \begin{itemize}
        \item Detect absence of main \texttt{.bin} file
        \item Detect presence of corrupted \texttt{.tmp}
        \item Ignore the temp file
        \item Load the latest valid snapshot before $t_0$
        \item If none exist: execute cold start
        \item Avoid infinite restart loops
    \end{itemize}
\end{enumerate}
Success criterion:
\[
\text{Recovery time} < T_{recovery} = 30 \text{ seconds}
\]
No functional degradation after recovery.
\end{criterion}

\begin{remark}
Atomic snapshot writes avoid the torn write problem. Most modern file systems (ext4, XFS, NTFS, APFS) guarantee atomic rename, which this protocol relies on.
\end{remark}

\subsection{Silent Disk Corruption}

\begin{testcase}[Bit Rot and Storage Errors]
Detect and handle silent data corruption that can occur between snapshot write and read.
\end{testcase}

\begin{criterion}
The protocol must:
\begin{enumerate}
    \item Store verification metadata with each snapshot:
    \begin{itemize}
        \item SHA-256 hash
        \item Creation timestamp
        \item Serialization format version
        \item Optional CRC32 pre-check
    \end{itemize}
    \item On load:
    \begin{itemize}
        \item Verify CRC32 if available
        \item Verify full SHA-256
        \item On failure, mark snapshot corrupt and search for previous valid snapshot
    \end{itemize}
    \item Retention policy:
    \begin{itemize}
        \item Keep last $N = 5$ valid snapshots
        \item Allow recovery from $t_{-k}$ if $t_0$ is corrupt
    \end{itemize}
\end{enumerate}
Validation:
\begin{enumerate}
    \item Create valid snapshot at $t_0$
    \item Corrupt random bytes
    \item Attempt load
    \item Verify system:
    \begin{itemize}
        \item Detects corruption
        \item Emits \texttt{CorruptedSnapshotEvent}
        \item Falls back to previous valid snapshot
        \item Executes cold start if no valid snapshot exists
        \item Avoids infinite retry loops
    \end{itemize}
\end{enumerate}
\end{criterion}

\subsection{Disk Space Exhaustion}

\begin{testcase}[Insufficient Capacity Handling]
Validate behavior when storage is full during snapshot write.
\end{testcase}

\begin{criterion}
During snapshotting, the system must:
\begin{enumerate}
    \item Before writing, verify free space:
    \[
    \text{FreeSpace} \geq 2 \times \text{EstimatedSize}(\Xi_t)
    \]
    (factor 2 for temp + rename)
    \item If insufficient space:
    \begin{itemize}
        \item Emit \texttt{InsufficientStorageEvent}
        \item Do not attempt write
        \item Keep last valid snapshot
        \item Continue operation in memory until space is available
    \end{itemize}
    \item If failure occurs mid-write:
    \begin{itemize}
        \item Catch I/O exception
        \item Delete corrupted temp file
        \item Emit \texttt{SnapshotWriteFailedEvent}
        \item Preserve last valid snapshot
    \end{itemize}
\end{enumerate}
Validation:
\begin{enumerate}
    \item Create a limited storage volume
    \item Fill it, leaving insufficient space
    \item Attempt snapshot
    \item Verify graceful handling without crash
\end{enumerate}
\end{criterion}

\begin{remark}
Disk exhaustion is common in production. Design priorities: (1) do not corrupt valid snapshots, (2) degrade gracefully in memory, (3) provide clear telemetry, (4) recover automatically when space frees.
\end{remark}

\chapter{Hardware Parity and Fidelity Tests (Cross-Platform)}

Given heterogeneous targets (CPU, GPU, FPGA), add bit-consistency tests to ensure equivalent outputs across architectures.

\section{Bit-Consistency Tests}

\begin{testcase}[Multi-Architecture Equivalence]
Verify that critical algorithms produce consistent numerical results across hardware platforms (CPU, GPU, FPGA), within precision limits of each architecture.
\end{testcase}

\begin{criterion}
For each critical component (random generation, signatures, SDE integration), execute identical inputs on:
\begin{enumerate}
    \item CPU with IEEE 754 floating-point (64-bit)
    \item GPU with floating-point (32 or 64-bit)
    \item FPGA with fixed-point arithmetic (configurable precision)
\end{enumerate}
Relative difference must satisfy:
\[
\frac{\|x^{CPU} - x^{GPU}\|_2}{\|x^{CPU}\|_2} < \epsilon_{GPU}, \quad \frac{\|x^{CPU} - x^{FPGA}\|_2}{\|x^{CPU}\|_2} < \epsilon_{FPGA}
\]
where $\epsilon_{GPU} = 10^{-6}$ for 32-bit arithmetic and $\epsilon_{FPGA}$ is the quantization error of the lower-precision hardware.
\end{criterion}

\section{Numerical Drift Test}

\subsection{Branch D: Signatures on FPGA vs CPU}

\begin{testcase}[Fixed-Point Error Accumulation]
Compare Branch D signatures on FPGA (fixed-point) against CPU (64-bit floating-point).
\end{testcase}

\begin{criterion}[Drift Acceptance Criteria]
For a path $\gamma: [0,T] \to \mathbb{R}^d$:
\[
\text{Sig}^{(M)}(\gamma) = \left(1, \int_0^T d\gamma_{t_1}, \int_0^T \int_0^{t_1} d\gamma_{t_2} \otimes d\gamma_{t_1}, \ldots \right)
\]
For 10,000 iterations:
\begin{enumerate}
    \item Compute CPU signature: $\text{Sig}^{CPU}_{10000}$
    \item Compute FPGA signature: $\text{Sig}^{FPGA}_{10000}$
    \item Compute accumulated divergence:
    \[
    \Delta_{acum} = \|\text{Sig}^{CPU}_{10000} - \text{Sig}^{FPGA}_{10000}\|_{\infty}
    \]
    \item \textbf{Primary criterion:}
    \[
    \Delta_{acum} \leq N \cdot \epsilon_{quant}
    \]
    where $N = 10000$ and $\epsilon_{quant} = 2^{-n+1}$ for $n$-bit fixed point

    \item \textbf{Secondary topological preservation:}
    \begin{itemize}
        \item Norm preservation:
        \[
        \frac{|\|\text{Sig}^{FPGA}_{10000}\|_2 - \|\text{Sig}^{CPU}_{10000}\|_2|}{\|\text{Sig}^{CPU}_{10000}\|_2} < \tau_{norm}
        \]
        with $\tau_{norm} = 0.01$
        \item Sign preservation:
        \[
        \text{sgn}(s_i^{(k),CPU}) = \text{sgn}(s_i^{(k),FPGA}) \quad \forall i,k
        \]
        \item Angular distance:
        \[
        \cos(\theta) = \frac{\langle \text{Sig}^{CPU}_{10000}, \text{Sig}^{FPGA}_{10000} \rangle}{\|\text{Sig}^{CPU}_{10000}\|_2 \cdot \|\text{Sig}^{FPGA}_{10000}\|_2}
        \]
        Require $\cos(\theta) > 0.9999$ (angular deviation $< 0.81^\circ$)
        \item Relative error per level:
        \[
        \frac{\|\text{Sig}^{CPU,(k)}_{10000} - \text{Sig}^{FPGA,(k)}_{10000}\|_2}{\|\text{Sig}^{CPU,(k)}_{10000}\|_2} < \tau_k
        \]
        where $\tau_k = 0.05 \cdot k$
    \end{itemize}
\end{enumerate}
\end{criterion}

\begin{remark}
Fixed-point drift is inevitable due to repeated truncation. The primary criterion bounds cumulative error by the worst-case independent error accumulation. For Branch D, qualitative properties must also be preserved: orientation, angular similarity, and low-level tensor accuracy.
\end{remark}

\begin{remark}
For FPGA with $n = 32$ bits (Q16.16), $\epsilon_{quant} = 2^{-15} \approx 3.05 \times 10^{-5}$. After 10,000 iterations, the primary criterion allows $\Delta_{acum} \leq 0.305$, while topological criteria enforce tighter constraints:
\begin{itemize}
    \item Relative norm error $< 1\%$
    \item Angular deviation $< 1^\circ$
    \item Level-1 error $< 5\%$
\end{itemize}
\end{remark}

\subsection{Deterministic Reproducibility}

\begin{testcase}[Controlled Seed Initialization]
Guarantee that, given the same pseudo-random seed, all platforms (CPU, GPU, FPGA) produce the same state sequence.
\end{testcase}

\begin{criterion}
Set deterministic seed $s_0$ and run 1,000 simulation steps on each platform. Verify:
\[
\{X_t^{CPU}\}_{t=1}^{1000} = \{X_t^{GPU}\}_{t=1}^{1000} = \{X_t^{FPGA}\}_{t=1}^{1000}
\]
Equality must be bit-for-bit for platforms with the same representation (CPU and GPU floating-point). For FPGA, compare after converting fixed-point to floating-point.
\end{criterion}

\section{Latency and Throughput Validation}

\begin{testcase}[Cross-Platform Performance Benchmark]
Measure execution time and throughput for each predictor branch on all architectures to identify bottlenecks and validate heterogeneous acceleration.
\end{testcase}

\begin{criterion}
For a batch of $N = 1000$ predictions:
\begin{align*}
T_{CPU} &= \text{total time on CPU} \\
T_{GPU} &= \text{total time on GPU} \\
T_{FPGA} &= \text{total time on FPGA}
\end{align*}
Expected:
\begin{itemize}
    \item GPU outperforms CPU for massively parallel operations: $T_{GPU} < 0.3 \cdot T_{CPU}$
    \item FPGA outperforms CPU for deterministic low-latency operations: $T_{FPGA} < 0.1 \cdot T_{CPU}$
\end{itemize}
Throughput:
\[
\text{Throughput} = \frac{N}{T} \quad \text{[predictions/second]}
\]
\end{criterion}

\chapter{Final Validation Protocol (Causality)}

This is the definitive predictive test before any deployment.

\section{Generalization}

\begin{testcase}[Rolling Walk-Forward]
Zero look-ahead bias. Training uses only data strictly prior to the test horizon.
\end{testcase}

\begin{criterion}
Split the dataset into rolling windows:
\[
\mathcal{D} = \{(t_1, y_1), \ldots, (t_N, y_N)\}
\]
For each test window $\mathcal{T}_k = \{t_{n_k}, \ldots, t_{n_k+w}\}$:
\begin{enumerate}
    \item Train only with $\mathcal{D}_{\text{train}}^k = \{(t_i, y_i) : t_i < t_{n_k}\}$
    \item Predict on $\mathcal{T}_k$
    \item Advance window: $k \to k+1$
    \item Aggregate out-of-sample metrics (RMSE, MAE, Sharpe ratio)
\end{enumerate}
The system must ensure that no future data is used to train the model at time $t$.
\end{criterion}

\section{Meta-Optimization Efficiency}

\begin{testcase}[Bayesian Optimization]
Iterative improvement of Expected Improvement on generalization error compared to random search.
\end{testcase}

\begin{criterion}
Use a Gaussian Process surrogate model for the hyperparameter space $\Theta = \{\alpha, \beta, \gamma, \ldots\}$. For $n$ optimization iterations:
\begin{enumerate}
    \item Random search: $\theta_i \sim \text{Uniform}(\Theta)$
    \item Bayesian optimization: $\theta_i = \arg\max_{\theta} \text{EI}(\theta | \mathcal{D}_{1:i-1})$
\end{enumerate}
Acceptance criterion:
\[
\min_{i \leq n} \mathcal{L}(\theta_i^{\text{BO}}) < \min_{i \leq n} \mathcal{L}(\theta_i^{\text{random}})
\]
where $\mathcal{L}$ is validation loss. Improvement must be significant (p-value $< 0.05$ in Mann-Whitney test).
\end{criterion}

\section{Temporal Integrity}

\begin{testcase}[Staleness Metric (TTL)]
Cancel JKO update if target signal delay $y_{target}$ exceeds $\Delta_{max}$.
\end{testcase}

\begin{criterion}
Define time-to-live:
\[
\text{TTL}(y_t) = t_{current} - t
\]
If:
\[
\text{TTL}(y_t) > \Delta_{max}
\]
The system must:
\begin{enumerate}
    \item Discard the signal
    \item NOT perform JKO update
    \item Emit \texttt{StaleDataEvent}
    \item Log the rejected timestamp
\end{enumerate}
Typical value: $\Delta_{max} = 5$ seconds for high-frequency systems.
\end{criterion}

\subsection{Lag Injection Test}

The staleness policy prevents operating with obsolete weights. This test validates behavior under extreme latency.

\begin{testcase}[Artificial Signal Delay]
Artificially delay $y_{target}$ beyond $\Delta_{max}$ and validate that degraded inference mode is activated and optimal transport is suspended.
\end{testcase}

\begin{criterion}
Validation procedure:
\begin{enumerate}
    \item Configure system with $\Delta_{max} = 5$ seconds
    \item Generate real-time data stream with correct timestamps
    \item Inject delayed signal $\tilde{y}_t$:
    \[
    \text{TTL}(\tilde{y}_t) = t_{current} - t = \Delta_{max} + \delta
    \]
    where $\delta > 0$ (typically $\delta = 1$ second)
    \item Verify the sequence:
    \begin{itemize}
        \item Detect $\text{TTL}(\tilde{y}_t) > \Delta_{max}$
        \item Activate \texttt{DegradedInferenceMode} = \texttt{True}
        \item Suspend JKO transport:
        \[
        \text{JKO\_update}(\rho^{n+1}) \to \text{SUSPENDED}
        \]
        \item Freeze orchestrator weights at last valid value: $\{w_i\}_{frozen}$
        \item Emit events:
        \begin{itemize}
            \item \texttt{StaleDataEvent} with metadata $(t, t_{current}, \text{TTL})$
            \item \texttt{DegradedInferenceModeActivated}
        \end{itemize}
        \item Log lag duration for post-mortem analysis
        \item Continue predictions using frozen weights only
    \end{itemize}
    \item Verify recovery when fresh signals satisfy $\text{TTL}(y_t) < 0.8 \cdot \Delta_{max}$:
    \begin{itemize}
        \item Deactivate \texttt{DegradedInferenceMode}
        \item Resume JKO transport
        \item Emit \texttt{NormalOperationRestoredEvent}
    \end{itemize}
\end{enumerate}
Acceptance criteria:
\begin{itemize}
    \item Detection time $< 100$ ms from receiving $\tilde{y}_t$
    \item No JKO iteration executed with stale data
    \item No crash or undefined state
    \item Predictions continue (degraded) with last valid configuration
\end{itemize}
\end{criterion}

\begin{remark}
Degraded inference mode is critical in high-frequency systems. Operating with stale weights is equivalent to optimizing the wrong objective. It is better to operate with consistent static weights than with weights optimized on obsolete data.
\end{remark}

\chapter{Edge Cases and Operational Limits}

This chapter documents boundary conditions for theoretical and operational behavior.

\section{CUSUM: Adaptive Dynamic Threshold}

\begin{testcase}[Volatility Regime Adaptation]
Validate that the CUSUM threshold adapts correctly to low and high volatility via $h = k \cdot \sigma_{resid}$.
\end{testcase}

\begin{criterion}
CUSUM uses dynamic threshold:
\[
 h_t = k \cdot \sigma_{resid, t}
\]
where $k \in [3, 5]$ and $\sigma_{resid, t}$ is rolling residual standard deviation.

Validation:
\begin{enumerate}
    \item \textbf{Low volatility:}
    \begin{itemize}
        \item Generate signal with $\sigma_{true} = 0.01$
        \item Estimate $\hat{\sigma}_{resid} \approx 0.01$
        \item Verify $h_t = k \cdot 0.01$
        \item Inject small drift $\Delta \mu = 0.05$
        \item Detector should trigger when $G_t^+ > h_t$
    \end{itemize}
    \item \textbf{High volatility:}
    \begin{itemize}
        \item Generate signal with $\sigma_{true} = 0.50$
        \item Estimate $\hat{\sigma}_{resid} \approx 0.50$
        \item Verify $h_t = k \cdot 0.50$
        \item Inject the same drift $\Delta \mu = 0.05$
        \item Detector should NOT trigger
        \item Inject larger drift $\Delta \mu = 2.0$
        \item Detector should trigger
    \end{itemize}
    \item \textbf{Transition:}
    \begin{itemize}
        \item Simulate transition from low to high volatility
        \item Verify $h_t$ updates smoothly (rolling window)
        \item No spurious activations during transition
    \end{itemize}
\end{enumerate}
Acceptance criterion:
\[
\frac{h_{high}}{h_{low}} = \frac{\sigma_{high}}{\sigma_{low}} \pm 0.1
\]
Threshold ratio must match volatility ratio within 10\% tolerance.
\end{criterion}

\section{Orchestrator: Maximum Entropy Convergence}

\begin{testcase}[Uniform Weights under Total Uncertainty]
Confirm that the system converges to uniform weights $w = [0.25, 0.25, 0.25, 0.25]$ when Sinkhorn regularization tends to infinity ($\varepsilon \to \infty$), representing maximum uncertainty.
\end{testcase}

\begin{criterion}
In Sinkhorn, entropy regularization $\varepsilon$ controls smoothing:
\[
\min_{\pi \in \Pi(\mu, \nu)} \left\{ \langle C, \pi \rangle - \varepsilon H(\pi) \right\}
\]
with $H(\pi) = -\sum_{ij} \pi_{ij} \log \pi_{ij}$.

Limit behavior:
\begin{enumerate}
    \item $\varepsilon \to 0$: deterministic optimal transport
    \item $\varepsilon \to \infty$: maximum entropy dispersion
\end{enumerate}
Validation:
\begin{enumerate}
    \item Configure four branches $(A, B, C, D)$
    \item Run with $\varepsilon_k = 10^k$ for $k \in \{0, 1, 2, 3, 4\}$
    \item Record weights $\{w_A^k, w_B^k, w_C^k, w_D^k\}$
    \item Verify convergence:
    \[
    \lim_{\varepsilon \to \infty} \{w_i\} = \left\{\frac{1}{4}, \frac{1}{4}, \frac{1}{4}, \frac{1}{4}\right\}
    \]
    \item Numerical criterion for $\varepsilon = 10^4$:
    \[
    \max_{i \in \{A,B,C,D\}} |w_i - 0.25| < 0.01
    \]
\end{enumerate}
Interpretation:
\begin{itemize}
    \item High $\varepsilon$ implies indistinguishable branches, thus uniform weights
    \item Maximum entropy principle (Jaynes)
\end{itemize}
\end{criterion}

\begin{remark}
This test validates that the orchestrator respects fundamental information theory. Under total uncertainty it must not favor any branch.
\end{remark}

\section{Branch D: Time Reparametrization Invariance}

\begin{testcase}[Signature Invariance to Time Warping]
Test a signal and its time-stretched variants; the rough path signature must be identical under strictly increasing reparametrizations.
\end{testcase}

\begin{criterion}
Signature invariance:
\[
\text{Sig}(\gamma) = \text{Sig}(\gamma \circ \phi)
\]
for any strictly increasing $\phi: [0,1] \to [0,1]$ with $\phi(0)=0$, $\phi(1)=1$.

Validation:
\begin{enumerate}
    \item Generate reference signal $\gamma(t)$ for $t \in [0,1]$
    \item Compute truncated signature $S_0 = \text{Sig}^{(M)}(\gamma)$
    \item Apply nonlinear reparametrizations:
    \[
    \phi_1(t) = t^2
    \]
    \[
    \phi_2(t) = \sqrt{t}
    \]
    \[
    \phi_3(t) = \frac{1}{2}(1 - \cos(\pi t))
    \]
    \item For each, compute:
    \[
    \gamma_i(t) = \gamma(\phi_i(t)), \quad S_i = \text{Sig}^{(M)}(\gamma_i)
    \]
    \item Verify:
    \[
    \|S_i - S_0\|_2 < \epsilon_{inv} \quad \forall i \in \{1, 2, 3\}
    \]
    with $\epsilon_{inv} = 10^{-8}$
\end{enumerate}
Negative control:
\begin{enumerate}
    \item Apply a non-monotone reparametrization $\psi(t) = t^2 - 0.5t$
    \item Verify $\text{Sig}(\gamma \circ \psi) \neq \text{Sig}(\gamma)$
\end{enumerate}
\end{criterion}

\begin{remark}
Signature invariance to time warping captures intrinsic path geometry regardless of execution speed. In finance, this means the signature captures the shape of a price move whether it occurred in 1 minute or 1 hour.
\end{remark}

\section{Edge Case Summary Table}

\begin{table}[h]
\centering
\caption{Limit Scenarios and Edge Cases}
\begin{tabular}{@{}llp{6cm}@{}}
\toprule
\textbf{Module} & \textbf{Test Scenario} & \textbf{Purpose} \\ \midrule
CUSUM & Dynamic threshold & Validate $h = k \cdot \sigma_{resid}$ adapts to low/high volatility \\ \addlinespace
Orchestrator & Maximum entropy ($\varepsilon \to \infty$) & Confirm convergence to uniform weights $[0.25, 0.25, 0.25, 0.25]$ \\ \addlinespace
Branch D & Time reparametrization invariance & Signature remains identical under time warping ($< 10^{-8}$ error) \\ \bottomrule
\end{tabular}
\end{table}

\chapter{Acceptance Criteria Summary}

\begin{table}[h]
\centering
\caption{Global Validation Table}
\begin{tabular}{@{}lll@{}}
\toprule
\textbf{Test} & \textbf{Method} & \textbf{Acceptance Criterion} \\ \midrule
Generalization & Rolling walk-forward & No look-ahead bias; training uses $t < t_{test}$ \\ \addlinespace
Meta-optimization efficiency & Bayesian optimization (GP) & Expected Improvement beats random search \\ \addlinespace
Temporal integrity & TTL staleness metric & Cancel JKO update if TTL$(y) > \Delta_{max}$ \\ \addlinespace
Lag injection & Artificial delay $> \Delta_{max}$ & Immediate degraded mode; JKO suspended \\ \addlinespace
Nyquist soft limit & Multifractal aliasing (SIA) & Freeze topological branch before $H$ error exceeds 10\% \\ \addlinespace
HJB-DGM stability & Gradient monitoring & Clip and reduce $\eta$ under sustained explosion \\ \addlinespace
Viscosity solutions & Comparison principle & Error $< 5\%$ vs reference; sub/supersolution verified \\ \addlinespace
Mode collapse & Training entropy & Variance ratio in $[0.3, 1.2]$ and proportional to terminal variance \\ \addlinespace
I/O atomicity & Write interruption & Safe cold start; recovery $< 30$ s \\ \addlinespace
Silent corruption & Bit rot detection & SHA-256 check with fallback to previous snapshot \\ \addlinespace
Dynamic CUSUM & Volatility adaptation & Threshold ratio follows volatility ratio within 10\% \\ \addlinespace
Maximum entropy & Sinkhorn $\varepsilon \to \infty$ & Uniform weights within 1\% \\ \addlinespace
Time invariance & Path reparametrization & Signature error $< 10^{-8}$ \\ \addlinespace
Bit consistency & Cross-platform tests & CPU/GPU/FPGA equivalence within quantization limits \\ \addlinespace
Numerical drift & Branch D fixed-point & Cumulative divergence $\leq N \cdot \epsilon_{quant}$ after 10k iterations \\ \bottomrule
\end{tabular}
\end{table}

\section{Final Considerations}

All protocols here are language-agnostic and based solely on the mathematical and algorithmic foundations of the Universal Stochastic Predictor.

\begin{remark}
The full test suite must run before any production deployment. Results must be documented in a validation report including:
\begin{itemize}
    \item Executive summary of all tests executed
    \item Numeric metrics for each criterion
    \item Convergence plots and distributions
    \item Failure case analysis (if any)
    \item Parameter calibration recommendations
\end{itemize}
\end{remark}

\begin{remark}
This protocol is a living document that must evolve with the system. Any architectural change in theory must be reflected in new test cases or updates to existing ones.
\end{remark}

\end{document}
