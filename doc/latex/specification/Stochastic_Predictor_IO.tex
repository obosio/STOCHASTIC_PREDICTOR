\documentclass{article}
\usepackage[utf8]{inputenc}
\usepackage{amsmath}
\usepackage{amsfonts}
\usepackage{geometry}
\usepackage{array}
\usepackage[english]{babel}
\usepackage[hidelinks]{hyperref}

\geometry{a4paper, margin=1in}

\title{Input/Output Interface Specification - Universal Predictor System}
\author{Systems Architecture}
\date{\today}

\begin{document}

\maketitle

\tableofcontents
\newpage

\section{Executive Summary}
This document defines the abstract input/output (I/O) interface for the Universal Predictor System, independent of the concrete implementation (Python/JAX, C++, Rust, FPGA). It describes the configuration vectors needed to instantiate the system, the runtime data flow, and the structure of the output signals and telemetry.

\section{Configuration Vector (Hyper-Inputs)}
The system is initialized with a configuration vector $\Lambda$ that defines module topology and sensitivity. These parameters are typically static during an operating session or tuned by an external meta-optimizer.

\begin{table}[h]
\centering
\begin{tabular}{|l|c|l|}
\hline
\textbf{Parameter} & \textbf{Symbol} & \textbf{Functional Description} \\
\hline
Entropic regularization & $\epsilon$ & Mass transport smoothing in the JKO orchestrator (Sinkhorn). \\
Learning rate & $\tau$ & Adaptation speed of weights $\rho$ under energy gradients. \\
Signature depth & $L$ & Truncation order of the log-signature (Kernel D - topological). \\
WTMM memory & $N_{buf}$ & Sliding window size for singularity estimation. \\
Besov cone & $C_{besov}$ & Influence radius for wavelet maxima tracking. \\
Holder threshold & $H_{min}$ & Critical regularity that triggers the circuit breaker. \\
CUSUM threshold & $h$ & Accumulated deviation level that triggers weight reset. \\
CUSUM slack & $k$ & Drift tolerance ("white noise") allowed without error accumulation. \\
Volatility memory & $\alpha$ & EMA decay rate to estimate error variance. \\
\hline
\end{tabular}
\caption{Hyperparameter vector $\Lambda$}
\end{table}

\section{Input Flow (Data Injection)}

\subsection{1. Calibration Phase (Bootstrapping)}
Initial state required before sequential operation.

\textbf{Input:} History $\mathcal{H} = \{y_{-T}, \dots, y_0\}$
\begin{itemize}
    \item \textbf{Structure:} Time series of vectors $\mathbf{y} \in \mathbb{R}^d$ or scalars $y \in \mathbb{R}$.
    \item \textbf{Purpose:}
    \begin{itemize}
        \item Initialize history-dependent kernels (e.g., Levy parameters).
        \item Stabilize initial orchestrator weights $\rho_0$.
        \item Fill the singularity buffer for the WTMM module.
    \end{itemize}
\end{itemize}

\subsection{2. Operational Phase (Online Stream)}
Step-by-step update cycle at time $t$.

\textbf{Input at step $t$:} Tuple $(y_t, y_{target}, \tau_{epoch})$
\begin{itemize}
    \item \textbf{Timestamp ($\tau_{epoch}$):} Absolute timestamp (Unix nanoseconds). Required for synchronization and latency checks in the staleness policy.
    \item \textbf{Current observation ($y_t$):}
    \begin{itemize}
        \item New data point available at $t$.
        \item Used to feed kernels ($K_A, K_B, K_C, K_D$) and generate predictions for $t+1$.
        \item \textbf{Domain error handling:} If $|y_t| > 20\sigma$ (relative to historical normalization), the point is classified as a catastrophic outlier. The system must discard the input, keep the inertial state, and emit a critical validation alert to protect kernels from numerical divergence.
        \item \textbf{Frozen signal detection:} If the stream injects the exact same value for $N_{freeze} \geq 5$ consecutive steps, the system must:
        \begin{enumerate}
            \item Compute the variance of the last $N_{freeze}$ values: $\text{Var}([y_{t-4}, y_{t-3}, y_{t-2}, y_{t-1}, y_t]) = 0$
            \item Identify this as sensor failure or data source corruption
            \item Emit \texttt{FrozenSignalAlarmEvent} with the event timestamp
            \item \textbf{Mathematical impact:} The Holder exponent in Branch D requires variability: $H_t = \lim_{s \to 0} \frac{\log |\gamma(t+s) - \gamma(t)|}{\log s}$. With a frozen signal, the numerator is zero, causing singularities or indeterminate values. This invalidates the multifractal spectrum. The system must:
            \begin{itemize}
                \item Freeze the topological branch (Kernel D) at the last valid value
                \item NOT update orchestrator weights (keep inertia)
                \item Activate degraded inference mode
                \item Continue predictions using Branches A, B, C only
            \end{itemize}
            \item \textbf{Recovery:} Once variance $> 0.1 \times \text{Var}_{historical}$ is detected for 2 consecutive steps, release the Kernel D lock and resume normal operation.
        \end{enumerate}
    \end{itemize}
    \item \textbf{Inference grid (anti-aliasing input):}
    \begin{itemize}
        \item \textbf{Sampling frequency vs scales ($N_{buf}$):} To guarantee WTMM stability (Kernel D singularities), the data injection frequency must maintain sufficient density relative to the finest wavelet scales.
        \item \textit{Restriction:} A \textbf{minimum injection frequency} (Nyquist soft limit) is enforced based on $C_{besov}$. If event density falls below this threshold, the multifractal spectrum collapses and the system must freeze the topological branch update.
    \end{itemize}

    \item \textbf{Validation target ($y_{target}$):}
    \begin{itemize}
        \item The "real" value corresponding to the prediction generated at the previous step ($t-1$).
        \item Typically $y_{target} \equiv y_t$ for causal one-step-ahead prediction.
        \item Used to compute error $e_t = y_{target} - \hat{y}_{t|t-1}$ and the energy gradient $\nabla E$ for JKO transport.
    \end{itemize}

    \item \textbf{Staleness policy:}
    \begin{itemize}
        \item \textbf{Time-to-live (TTL):} Parameter $\Delta_{max}$.
        \item \textbf{Violation behavior:} If the delay of $y_{target}$ exceeds $\Delta_{max}$, the JKO update is canceled.
        \item \textbf{Integrity signal:} The system must emit a persistent \textit{degraded inference} flag ("stale weights"). This alerts the executor that, although the prediction $\hat{y}$ is still produced, the weights $\rho$ are stale and risk is no longer optimized geometrically.
    \end{itemize}
\end{itemize}

\section{Security Policies in the I/O Layer (Credentials)}

The Stochastic Predictor design requires high-frequency ingestion against external market infrastructure (e.g., institutional WebSockets, brokers, or REST APIs). Access to these systems introduces critical vulnerability vectors.

\vspace{0.5em}

\noindent \textbf{Secure environment injection}

Hardcoding tokens, API keys, database secrets, or connection credentials in any future source module (e.g., \texttt{io/streams.py}) is strictly forbidden. Every implementation \textbf{must} apply an environment injection pattern, reading credentials at runtime from OS variables or local \texttt{.env} files.

\vspace{0.5em}

\noindent \textbf{Version control exclusion}

The resulting implementation repository must include explicit exclusion rules in version control (e.g., enforce \texttt{*.env} in \texttt{.gitignore}) to ensure no secret is exposed.

\section{System Outputs}

\subsection{1. Control signal (Prediction signal)}
Primary output for decision-making.

\textbf{Output:} $\hat{y}_{t+1}$
\begin{itemize}
    \item \textbf{Description:} Estimate of the expected process value for the next instant.
    \item \textbf{Inference grid (output quantization):}
    \begin{itemize}
        \item Output $\hat{y}_{t+1}$ is delivered in normalized space (Z-score) consistent with input $y_t$.
        \item The actor/executor applies inverse normalization using rolling-window statistics if an absolute price is required.
    \end{itemize}
    \item \textbf{Composition:} Convex combination of base kernels: $\hat{y}_{t+1} = \sum_{i \in \{A,B,C,D\}} \rho_i^{(t)} \cdot K_i(y_t)$.
\end{itemize}

\subsection{2. State telemetry}
Latent variables that describe system "health" and market regime.

\begin{itemize}
    \item \textbf{Risk state ($\mathbb{S}_{risk}$):}
    \begin{itemize}
        \item \textbf{Local Holder exponent ($H_t$):} Pointwise regularity measure. $H_t < 0.5$ indicates antipersistence/noise; $H_t < H_{min}$ indicates imminent crash/shock.

        \item \textbf{Empirical kurtosis ($\kappa_t$):} Fourth standardized moment of prediction residuals over a rolling window:
        \[
        \kappa_t = \frac{E[(e_t - \mu_e)^4]}{(\sigma_e)^4}
        \]
        where $e_t = y_{target} - \hat{y}_{t|t-1}$ are prediction residuals.
        \begin{itemize}
            \item \textbf{Purpose:} Validate the adaptive CUSUM threshold. Values $\kappa_t > 3$ indicate leptokurtic distributions (heavy tails), activating the logarithmic adjustment $h_t = k \cdot \sigma \cdot (1 + \ln(\kappa_t/3))$.
            \item \textbf{Interpretation:} $\kappa_t \approx 3$ (Gaussian), $\kappa_t \in [5, 10]$ (standard financial volatility regime), $\kappa_t > 15$ (crisis regime with frequent extreme events).
            \item \textbf{Alert:} If $\kappa_t > 20$ persistently, emit a warning of potential residual model failure or undetected systematic outliers.
        \end{itemize}

        \item \textbf{DGM predictor entropy ($H_{DGM}$):} Differential entropy of the neural value function $V_\theta(t,x)$ across the spatial domain:
        \[
        H_{DGM} = -\int p_V(v) \log p_V(v) \, dv
        \]
        where $p_V(v)$ is the empirical density of $V_\theta$ values evaluated on a domain grid.
        \begin{itemize}
            \item \textbf{Purpose:} Monitor the health of Branch B (HJB solution via Deep Galerkin Method) and detect mode collapse when the network predicts a constant or degenerate solution.
            \item \textbf{Collapse threshold:} Compare against the terminal condition entropy: $H_{DGM} \geq \gamma \cdot H[g]$ with $\gamma \in [0.5, 1.0]$. If the inequality is violated persistently (more than 10 consecutive steps), emit \texttt{ModeDegradationAlert}.
            \item \textbf{Corrective action:} Reduce the weight of Branch B in the orchestrator ($\rho_B \to 0$) and prioritize alternative branches until the DGM network is retrained or reinitialized.
            \item \textbf{Note:} This indicator is only relevant if Branch B is active ($\rho_B > 0.05$). For systems that do not use DGM, this field may be omitted or reported as \texttt{NaN}.
        \end{itemize}

        \item \textbf{CUSUM statistic ($G^+$):} Accumulated structural mismatch level.

        \item \textbf{Distance to collapse ($h_t - G^+$):} Safety margin before a forced model reset. Note: $h_t$ is now dynamic and depends on $\sigma_t$ and $\kappa_t$.

        \item \textbf{Residual free energy ($\mathcal{F}$):} Instant value of the JKO functional. It monitors whether the model is "stuck" in a stable local minimum or whether entropic regularization $\epsilon$ is too high, over-smoothing mass transport and diluting predictive power.
    \end{itemize}

    \item \textbf{Orchestrator state ($\rho$):}
    \begin{itemize}
        \item \textbf{Weight vector:} $[\rho_A, \rho_B, \rho_C, \rho_D]$ such that $\sum \rho = 1$.
        \item Indicates which "physics" currently dominates the market (jumps vs diffusion vs memory vs topology).
    \end{itemize}

    \item \textbf{Stochastic health-check:}
    \begin{itemize}
        \item \textbf{Sinkhorn convergence (bool):} Indicates whether the mass transport algorithm converged within the maximum iteration count.
        \item \textit{True:} Exact Wasserstein distance. \textit{False:} Sub-optimal approximation (numerical precision alert).
    \end{itemize}

    \item \textbf{Operational flags (mode and circuit breakers):}
    \begin{itemize}
        \item \textbf{Base operation mode:}
        \begin{itemize}
            \item \textit{Standard (MSE):} Normal operation under local Gaussian assumptions.
            \item \textit{Robust (Huber):} Defensive operation triggered by singularities ($H_t < H_{min}$) or high volatility.
        \end{itemize}

        \item \textbf{Degraded inference mode:} Critical boolean flag for temporal quality monitoring:
        \begin{itemize}
            \item \textbf{Activation condition:} It activates when the time-to-live (TTL) of $y_{target}$ exceeds the maximum threshold:
            \[
            \text{TTL}(y_{target}) = t_{current} - t_{signal} > \Delta_{max}
            \]
            \item \textbf{Operational implications:}
            \begin{enumerate}
                \item JKO transport update is suspended immediately
                \item Weights $\rho$ freeze at their last valid value (inertial mode)
                \item Predictions $\hat{y}_{t+1}$ continue, but are sub-optimal because they do not reflect the true market state
                \item Risk is no longer optimized geometrically
            \end{enumerate}
            \item \textbf{Executor signaling:} This flag must explicitly warn that:
            \begin{itemize}
                \item Current predictions have \textit{degraded confidence}
                \item Weights are stale
                \item Exposure should be reduced or a conservative mode should be used
                \item The system operates in "survival mode" until fresh data flow is restored
            \end{itemize}
            \item \textbf{Recovery:} The flag is automatically cleared when a fresh signal arrives with $\text{TTL}(y_{target}) < 0.8 \cdot \Delta_{max}$ (hysteresis threshold to avoid oscillation). At that moment, JKO transport resumes and \texttt{NormalOperationRestoredEvent} is emitted.
        \end{itemize}

        \item \textbf{Emergency mode (singularity fallback):} Flag indicating whether emergency mode was triggered by critical singularity ($H_t < H_{min}$), forcing $w_D \to 1.0$ and switching to the Huber metric.

        \item \textbf{Regime change detected:} Flag indicating whether CUSUM detected a regime change at the last step, with entropy reset to a uniform distribution.
    \end{itemize}
\end{itemize}

\section{Abstract I/O Diagram}

\[
\boxed{\text{Environment}} \xrightarrow{(y_t, y_{target})} \boxed{\text{Universal Predictor}(\Lambda)} \xrightarrow{(\hat{y}_{t+1}, \mathbb{S}_{risk}, \rho)} \boxed{\text{Actor / Executor}}
\]

\subsection{Internal Process Cycle}
\begin{enumerate}
    \item \textbf{Ingestion:} Receive $y_t$. Update local history.
    \item \textbf{Singularity analysis:} Compute $H_t$ using WTMM on the recent window.
    \item \textbf{Quality control (CUSUM):}
        \begin{itemize}
            \item Compute error $e_t$ using $y_{target}$ and stored prediction $\hat{y}_{t|t-1}$.
            \item Update drift accumulator $G^+$.
            \item If $G^+ > h$ or $H_t < H_{min}$ $\rightarrow$ emit reset/alert signal.
        \end{itemize}
    \item \textbf{Transport (JKO):}
        \begin{itemize}
            \item Compute energy gradient $\nabla E$ based on $e_t$.
            \item Transport probability mass $\rho_{t-1} \to \rho_t$ (Sinkhorn).
        \end{itemize}
    \item \textbf{Projection:}
        \begin{itemize}
            \item Execute kernels $K_i(y_t)$ to obtain components.
            \item Aggregate components using new weights $\rho_t$ to obtain $\hat{y}_{t+1}$.
        \end{itemize}
\end{enumerate}

\section{Persistence (Snapshotting)}
To ensure operational continuity, the system must be able to serialize its full internal state $\Sigma_t$ at any time $t$.

\[
\Sigma_t = \{ \rho_t, G^+_t, \sigma^2_{ema}, \kappa_t, H_{DGM}, \text{Flags}, \text{WTMMBuffer}, \text{KernelsState} \}
\]

where:
\begin{itemize}
    \item $\kappa_t$: Rolling empirical kurtosis of prediction errors (window size = 252).
    \item $H_{DGM}$: Differential entropy of the DGM predictor (for mode collapse detection).
    \item \texttt{Flags}: Boolean flags including \texttt{DegradedInferenceMode}, \texttt{EmergencyMode}, and \texttt{RegimeChangeDetected}.
\end{itemize}

The \texttt{KernelsState} structure must be segmented into independent sub-blocks (K-blocks) to allow modular or partial updates:
\[
\text{KernelsState} = \{ S_A (\text{Levy}), S_B (\text{PDE}), S_C (\text{Memory}), S_D (\text{Topology}) \}
\]

The restore operation $Load(\Sigma_t)$ must allow the flow to resume at $t+1$ without recalibration over history $\mathcal{H}$. Correct restoration of $\kappa_t$ and $H_{DGM}$ is critical to preserve anomaly detection and mode collapse sensitivity after a restart.

\subsection{Atomic and Verified Snapshotting Protocol}
Binary serialization formats (e.g., Protocol Buffers, MessagePack) are required instead of text (JSON/XML) to minimize I/O latency for critical hot-start operations.

\begin{itemize}
    \item \textbf{Mandatory integrity checksum:} Because dense binary formats are used, a single-bit error in kernel matrices or the WTMM buffer could trigger system collapse or undefined behavior. Therefore, $\Sigma_t$ must include a robust validation hash (e.g., SHA-256 or CRC32c) at the end of the block.
    \item \textbf{Pre-injection validation:} The restore routine $Load(\Sigma_t)$ must recompute and verify this hash \textit{before} injecting the state into operational memory. If validation fails, the snapshot must be discarded and the system must restart in cold-start mode (history reload).
\end{itemize}

\section{Telemetry and Deterministic Logging}

\subsection{Non-Blocking Telemetry}
Telemetry emission must be decoupled from JAX/XLA execution. Compute threads must never block on I/O. Telemetry buffers are enqueued in a non-blocking structure and consumed by a separate process or thread.

\subsection{Parity Audit Hashes}
For hardware parity validation, log SHA-256 hashes of the weight vector $\rho$ and the OT cost at configurable intervals. Hash inputs must use canonical float64 serialization to guarantee deterministic parity across CPU and GPU.

\begin{itemize}
    \item Hash interval is configuration-driven.
    \item Logs are append-only and immutable.
    \item No raw signals or secrets are emitted in telemetry.
\end{itemize}

\end{document}
