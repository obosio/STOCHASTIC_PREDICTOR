\documentclass[11pt, a4paper]{report}

% --- PREAMBLE ---
\usepackage[a4paper, top=2.5cm, bottom=2.5cm, left=2cm, right=2cm]{geometry}
\usepackage{fontspec}
\usepackage{amsmath}
\usepackage{amssymb}
\usepackage{listings}
\usepackage{xcolor}
\usepackage[hidelinks]{hyperref}
\usepackage{graphicx}
\usepackage{booktabs}
\usepackage{array}

\usepackage[english]{babel}

% Code highlighting
\definecolor{codegreen}{rgb}{0,0.6,0}
\definecolor{codegray}{rgb}{0.5,0.5,0.5}
\definecolor{codepurple}{rgb}{0.58,0,0.82}
\definecolor{backcolour}{rgb}{0.95,0.95,0.92}

\lstdefinestyle{mystyle}{
    backgroundcolor=\color{backcolour},
    commentstyle=\color{codegreen},
    keywordstyle=\color{magenta},
    numberstyle=\tiny\color{codegray},
    stringstyle=\color{codepurple},
    basicstyle=\ttfamily\footnotesize,
    breakatwhitespace=false,
    breaklines=true,
    captionpos=b,
    keepspaces=true,
    numbers=left,
    numbersep=5pt,
    showspaces=false,
    showstringspaces=false,
    showtabs=false,
    tabsize=2,
    frame=single
}

\lstset{style=mystyle}

\title{\textbf{Universal Stochastic Predictor \\ Phase 4: IO Layer Initiation}}
\author{Implementation Team}
\date{February 19, 2026}

\begin{document}

\maketitle

\tableofcontents

\chapter{Phase 4: IO Layer Initiation Overview}

Phase 4 introduces the asynchronous I/O layer for snapshots, streaming, and telemetry export. The primary design goal is to preserve JAX/XLA throughput by decoupling compute from disk or network latency.

\section{Scope}

Phase 4 covers:
\begin{itemize}
    \item \textbf{Telemetry Buffering}: Non-blocking emission of telemetry snapshots
    \item \textbf{Deterministic Logging}: Hash-based parity checks for CPU/GPU validation
    \item \textbf{Snapshot Strategy}: Atomic persistence of predictor state
\end{itemize}

\section{Design Principles}

\begin{itemize}
    \item \textbf{No Compute Stalls}: JAX compute threads never block on I/O
    \item \textbf{Determinism}: Logs capture reproducible hashes instead of raw state dumps
    \item \textbf{Security}: No raw signals or secrets in logs
    \item \textbf{Configurability}: Logging intervals and destinations injected via config
\end{itemize}

\chapter{Telemetry Abstraction}

\section{TelemetryBuffer Emission}

The JKO orchestrator should emit a \texttt{TelemetryBuffer} at the end of each step. This buffer is consumed by a dedicated process outside the JAX execution thread.

\begin{itemize}
    \item The buffer contains summary metrics (CUSUM, entropy, regime flags, OT cost).
    \item The compute path only enqueues the buffer and continues.
    \item The consumer is responsible for serialization and persistence.
\end{itemize}

\chapter{Deterministic Logging}

\section{Hash-Based Parity Checks}

For hardware parity audits, the logger records SHA-256 hashes of the weight vector $\rho$ and the OT cost at configurable intervals. This permits CPU/GPU parity validation without dumping VRAM data.

\begin{itemize}
    \item Hash interval configured per deployment.
    \item Hashes derived from canonical float64 serialization.
    \item Logs are append-only and immutable.
\end{itemize}

\chapter{Snapshot Strategy}

\section{Atomic Persistence}

Snapshots must be persisted atomically to prevent partial writes. The IO layer is responsible for:
\begin{itemize}
    \item Writing to temporary files and renaming atomically.
    \item Optional compression configured by policy.
    \item Coordinating snapshot cadence with telemetry output.
\end{itemize}

\chapter{Compliance Checklist}

\begin{itemize}
    \item \textbf{No Compute Stalls}: All logging is asynchronous
    \item \textbf{Deterministic Hashing}: SHA-256 on $\rho$ and OT cost
    \item \textbf{Security}: No raw signals, VRAM dumps, or secrets
    \item \textbf{Config-Driven}: Intervals and destinations are injected
\end{itemize}

\chapter{Phase 4 Summary}

Phase 4 introduces a non-blocking I/O architecture that preserves deterministic compute while enabling telemetry, logging, and atomic snapshot persistence.

\end{document}
