\documentclass[11pt,a4paper]{article}
\usepackage[utf8]{inputenc}
\usepackage[english]{babel}
\usepackage{amsmath,amssymb,amsthm}
\usepackage{geometry}
\usepackage{enumitem}
\usepackage{fancyhdr}
\usepackage{booktabs}
\usepackage{listings}
\usepackage{xcolor}
\usepackage[hidelinks]{hyperref}
\usepackage{xurl}

% Custom hyperlink commands for file and document references
\newcommand{\filehref}[1]{\href{file:../../#1}{\texttt{#1}}}
\newcommand{\dochref}[2]{\href{../../pdf/specification/#1.pdf}{\texttt{#2}}}

\geometry{margin=1in}
\pagestyle{fancy}
\fancyhf{}
\lhead{USP Testing Infrastructure Implementation}
\rhead{v2.2.2 | 2026-02-22}
\cfoot{\thepage}

% Code listings
\lstset{
    language=bash,
    basicstyle=\ttfamily\small,
    breaklines=true,
    keywordstyle=\color{blue},
    commentstyle=\color{gray},
    stringstyle=\color{red},
    backgroundcolor=\color{lightgray!20},
    frame=single,
    numbers=left,
    numberstyle=\tiny\color{gray}
}

\title{\textbf{Universal Stochastic Predictor}\\[0.5em]
       \textbf{Testing Infrastructure Implementation}\\[0.5em]
       \large Quality Gates Framework with Decentralized Reporting}
\author{Development Team}
\date{Document Version: 2.2.2 \\ Last Updated: 2026-02-22 \\ Test Framework: v2.2.2}

\begin{document}

\maketitle
\thispagestyle{fancy}

\begin{abstract}
\noindent This document describes the \textbf{v2.2.2 implementation} of the Universal Stochastic Predictor's testing infrastructure. The system implements a fail-fast quality gate architecture with decentralized reporting:

\begin{itemize}[noitemsep,leftmargin=1.5em]
    \item \textbf{Quality Gates:} Two-stage pipeline (GATE 1: Dependencies, GATE 2: Code Quality)
    \item \textbf{Decentralized Reporting:} Each tool generates JSON, then auto-generates Markdown
    \item \textbf{3-Level Classification:} BLOCKING (won't execute) / ERROR (executes but buggy) / WARNING (style)
    \item \textbf{Relative Path Format:} All file paths relative to project root for portability
    \item \textbf{Audited Stages:} GATE 1 (dependency\_check.py) and GATE 2 (code\_lint.py) fully audited
    \item \textbf{Temporal Exit:} After GATE 2, test generation and pytest execution remain unaudited
\end{itemize}

\noindent Status: ✅ GATE 1 PASSED (0 blocking, 0 errors, 2 warnings) · ✅ GATE 2 PASSED (0 blocking, 0 errors, 47 lint warnings)
\end{abstract}

\tableofcontents
\newpage

% ============================================================================
\section{System Overview}
% ============================================================================

\subsection{Architecture Summary}

The v2.2.2 testing infrastructure comprises:

\begin{center}
\begin{tabular}{|l|l|l|}
\hline
\textbf{Component} & \textbf{Location} & \textbf{Purpose} \\
\hline
Orchestrator & \filehref{Test/run\_tests.py} & Entry point with quality gates \\
\hline
GATE 1 & \filehref{Test/scripts/dependency\_check.py} & Dependency validation (AUDITED) \\
\hline
GATE 2 & \filehref{Test/scripts/code\_lint.py} & Code quality check (AUDITED) \\
\hline
Reporting & \filehref{Test/framework/reports.py} & Markdown generation engine \\
\hline
Reports & \filehref{Test/reports/} & Markdown outputs (auto-generated) \\
\hline
Results & \filehref{Test/results/} & JSON intermediates (auto-generated) \\
\hline
\end{tabular}
\end{center}

\subsection{Current Status (v2.2.2 - Audited Stages Only)}

\begin{center}
\begin{tabular}{|l|l|l|}
\hline
\textbf{Stage} & \textbf{Status} & \textbf{Details} \\
\hline
GATE 1: Dependencies & ✅ AUDITED & 0 blocking, 0 errors, 2 warnings \\
\hline
GATE 2: Code Quality & ✅ AUDITED & 0 blocking, 0 errors, 47 lint warnings \\
\hline
Test Generation & ⏳ PENDING & Will audit scaffold next \\
\hline
pytest Execution & ⏳ PENDING & Will audit after test generation \\
\hline
\end{tabular}
\end{center}

\noindent After GATE 2, execution stops (temporal exit) until remaining stages are audited and brought into production.

% ============================================================================
\section{GATE 1: Dependency Validation}
% ============================================================================

\subsection{Location and Purpose}

\filehref{Test/scripts/dependency\_check.py} — Validates all Python imports against declared requirements.

\subsubsection{Execution Flow}

\begin{enumerate}
    \item Parse \texttt{Python/requirements.txt} and \texttt{Test/requirements.txt}
    \item Scan all \texttt{.py} files in \texttt{Python/} for \texttt{import} statements
    \item Flag any imports not in the requirements (MISSING)
    \item Flag any requirements not imported (UNUSED)
    \item Output: JSON payload and auto-generated Markdown report
\end{enumerate}

\subsubsection{Issue Classification}

Uses 3-level hierarchy with tool prefix \texttt{[dependency]}:

\begin{itemize}[noitemsep]
    \item \textbf{BLOCKING:} Missing critical dependency (breaks import)
    \item \textbf{ERROR:} Unused requirement (bloat)
    \item \textbf{WARNING:} Transitive dependency issue
\end{itemize}

\subsubsection{Output Format (JSON)}

\begin{lstlisting}[language=Python]
{
  "stage": "GATE 1",
  "tool": "dependency_check.py",
  "result": {
    "total_requirements": 22,
    "total_imports": 45,
    "summary": {"blocking": 0, "error": 0, "warning": 2},
    "issues": [
      {
        "level": "warning",
        "category": "[dependency]",
        "message": "Unused requirement: scipy",
        "file": null
      }
    ]
  }
}
\end{lstlisting}

\subsubsection{Latest Results}

\begin{center}
\begin{tabular}{|l|r|}
\hline
\textbf{Metric} & \textbf{Count} \\
\hline
Total Requirements & 22 \\
Total Imports & 45 \\
Blocking Issues & 0 \\
Errors & 0 \\
Warnings & 2 \\
\hline
\end{tabular}
\end{center}

% ============================================================================
\section{GATE 2: Code Quality Control}
% ============================================================================

\subsection{Location and Purpose}

\filehref{Test/scripts/code\_lint.py} — Unified code quality validator integrating black, isort, flake8, and mypy.

\subsubsection{Validation Pipeline}

\begin{enumerate}
    \item \textbf{Format Check (black):} Enforce PEP 8 formatting
    \item \textbf{Import Sort (isort):} Organize imports alphabetically
    \item \textbf{Lint Check (flake8):} Detect style violations and errors
    \item \textbf{Type Check (mypy):} Validate type annotations against runtime
\end{enumerate}

\subsubsection{Issue Classification}

Uses 3-level hierarchy with tool-specific prefixes:

\begin{itemize}[noitemsep]
    \item \textbf{BLOCKING:} Type mismatch incompatible with JAX operations
    \item \textbf{ERROR:} Syntax or import errors
    \item \textbf{WARNING:} Style or lint issues (code still executes)
\end{itemize}

Tool prefixes for filtering:
\begin{center}
\begin{tabular}{|l|l|}
\hline
\textbf{Prefix} & \textbf{Tool} \\
\hline
\texttt{[format]} & black formatting \\
\texttt{[isort]} & Import organization \\
\texttt{[lint]} & flake8 violations \\
\texttt{[type]} & mypy type checking \\
\hline
\end{tabular}
\end{center}

\subsubsection{Output Format (JSON)}

\begin{lstlisting}[language=Python]
{
  "stage": "GATE 2",
  "tool": "code_lint.py",
  "result": {
    "total_files": 27,
    "summary": {"blocking": 0, "error": 0, "warning": 47},
    "breakdown": [
      {
        "tool": "flake8",
        "issues": [
          {
            "level": "warning",
            "category": "[lint]",
            "message": "E501: Line too long (>88 chars)",
            "file": "Test/scripts/code_structure.py",
            "line": 42
          }
        ]
      }
    ]
  }
}
\end{lstlisting}

\subsubsection{Latest Results}

\begin{center}
\begin{tabular}{|l|r|}
\hline
\textbf{Metric} & \textbf{Count} \\
\hline
Total Files Scanned & 27 \\
Blocking Issues & 0 \\
Errors & 0 \\
Warnings & 47 \\
\hline
\end{tabular}
\end{center}

% ============================================================================
\section{Report Generation Framework}
% ============================================================================

\subsection{Decentralized Architecture}

Each GATE generates its own report in two formats:

\begin{enumerate}
    \item \textbf{JSON:} Structured payload for machine parsing
    \item \textbf{Markdown:} Human-readable report with sections
\end{enumerate}

Location: \filehref{Test/framework/reports.py} — Rendering engine for JSON-to-Markdown conversion.

\subsubsection{Pipeline}

\begin{lstlisting}[language=Python]
# Step 1: GATE 1 collects issues
gate1_issues = dependency_check.run()

# Step 2: GATE 1 writes JSON
save_json(gate1_issues, 'Test/results/dependency_check_last.json')

# Step 3: GATE 1 renders Markdown
render_report(gate1_issues, 'Test/reports/DEPENDENCY_CHECK_LAST.md')
\end{lstlisting}

\subsubsection{Markdown Sections}

Auto-generated reports include:

\begin{itemize}[noitemsep]
    \item \textbf{Header:} Gate name, timestamp, version
    \item \textbf{Summary:} Blocking/Error/Warning counts
    \item \textbf{Breakdown:} Per-tool results with issue details
    \item \textbf{Classification:} Hierarchical grouping by level
\end{itemize}

% ============================================================================
\section{Output Directory Structure}
% ============================================================================

\subsection{Artifacts Location}

\begin{lstlisting}[language=bash]
Test/
├── results/                           # JSON payloads (machine)
│   ├── dependency_check_last.json     # GATE 1 JSON
│   └── code_lint_last.json            # GATE 2 JSON
├── reports/                           # Markdown reports (human)
│   ├── DEPENDENCY_CHECK_LAST.md       # GATE 1 report
│   └── CODE_LINT_LAST.md              # GATE 2 report
└── run_tests.py                       # Orchestrator (cleans & runs)
\end{lstlisting}

\subsection{Cleanup Behavior}

\textbf{Policy:} Every execution of \texttt{run\_tests.py} removes all prior JSON and Markdown files.

\begin{lstlisting}[language=bash]
# Before GATE 1 starts, cleanup:
rm -f Test/results/*.json
rm -f Test/reports/*.md
\end{lstlisting}

This ensures:
\begin{itemize}[noitemsep]
    \item No stale reports from previous runs
    \item Each execution produces fresh snapshots
    \item Clear audit trail of "latest" state
\end{itemize}

% ============================================================================
\section{Coming Next: Remaining Stages}
% ============================================================================

The v2.2.2 release includes audited implementation for:

\begin{itemize}
    \item \textbf{GATE 1:} Dependency validation ✓
    \item \textbf{GATE 2:} Code quality control ✓
\end{itemize}

\noindent
Planned for future audits:

\begin{itemize}
    \item \textbf{GATE 3:} Test execution
    \item \textbf{GATE 4:} Performance profiling
    \item \textbf{GATE 5:} Security scanning
    \item \textbf{GATE 6:} Documentation validation
\end{itemize}

\noindent
Each remaining GATE will follow the same 3-level classification scheme and auto-generated Markdown reporting pattern established in v2.2.2.

% ============================================================================
\section{Usage}
% ============================================================================

\subsection{Running Quality Gates}

\begin{lstlisting}[language=bash]
# Execute both GATE 1 and GATE 2
python Test/run_tests.py

# Output:
# GATE 1: Dependency Validation -> JSON + Markdown
# GATE 2: Code Quality Control -> JSON + Markdown
# (Exits temporally after GATE 2, remaining gates pending)

# View reports
cat Test/reports/DEPENDENCY_CHECK_LAST.md
cat Test/reports/CODE_LINT_LAST.md
\end{lstlisting}

% ============================================================================
\section{Path Format Convention}
% ============================================================================

All file paths in reports are \textbf{relative to project root}:

\begin{itemize}[noitemsep]
    \item \texttt{Python/api/config.py} (relative)
    \item NOT \texttt{/absolute/path/to/config.py} (absolute)
\end{itemize}

This ensures portability and clarity across development environments.

% ============================================================================
\section{Conclusion}
% ============================================================================

The v2.2.2 quality gates framework establishes:

\begin{itemize}[noitemsep]
    \item \textbf{GATE 1 (Audited):} Dependency validation—22 packages, 45 imports, 2 warnings
    \item \textbf{GATE 2 (Audited):} Code quality control—27 files, 47 lint warnings, 0 blocking issues
    \item \textbf{Decentralized Reporting:} Each gate auto-generates JSON + Markdown
    \item \textbf{3-Level Classification:} BLOCKING, ERROR, WARNING across all gates
    \item \textbf{Relative Paths:} All reports use project-root-relative paths for portability
\end{itemize}

\noindent
Future releases will extend this framework to include GATE 3–6 (test execution, profiling, security, documentation) using the same patterns established in v2.2.2.

\vspace{2cm}

\begin{center}
Last Updated: February 22, 2026 \\
Infrastructure Version: 2.2.2 \\
Framework: pytest 7.3.0 + Quality Gates \\
Specification: v2.1.0-Quality Gates (Audited)
\end{center}

\end{document}
